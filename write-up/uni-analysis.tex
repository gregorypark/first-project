\documentclass[12pt]{article}
% Packages
\usepackage[margin=1in]{geometry}
\usepackage{graphicx}
\usepackage{amsmath}
\usepackage{booktabs}
\usepackage{setspace}
\usepackage{natbib}
\usepackage{hyperref}
\usepackage{float}

% Title Info
\title{Analyzing the Relationship Between Dominant Major and 
Median Earnings Within U.S. Universities}
\author{Gregory Park}
\date{\today}

\begin{document}

\maketitle
\onehalfspacing

\section{Introduction}

The return on investment for higher education is an 
important concern for students and policymakers. 
This paper addresses the following research question: 
\textit{Do universities with different dominant majors 
produce different earnings outcomes for their graduates?} 
Understanding this relationship can inform both 
student college choices and institutional policy decisions.

\section{Data and Methodology}

\subsection{Data Source}

I worked with data from the U.S. Department of Education's College Scorecard 
(Most Recent Institution-Level Data, updated April 2023), 
which provides institution-level information on student demographics, outcomes, and 
admissions. The sample includes data on approximately 6,000 
institutions in the United States.

\subsection{Key Variables}
My analysis focuses on two key variables:
\begin{itemize}
    \item \textbf{Dominant Major}: The academic major representing the largest 
    percentage of degrees earned at each institution.
    
    \item \textbf{Median Earnings}: The median earnings of graduates 10 years 
    after initial enrollment.
\end{itemize}

\subsection{Sample Selection}

I restricted the sample to predominantly bachelor's-degree granting 
universities with complete earnings data. I also  
excluding specialized institutions where a single major 
represented more than 60\% of degrees awarded. 
This yielded a final sample of roughly 1,500 universities.

\section{Results}
\subsection{Descriptive Statistics}

Table \ref{tab:descriptive} shows aggregate median earnings 
by dominant major. Transportation shows the highest 
median earnings (\$84k), followed by Social Sciences (\$73k) and 
Engineering (\$72k). Insitituions focused on Agriculture, 
Theology, and Liberal Arts have the lowest 
median earnings at below \$45k.

\subsection{Main Findings}

Figure \ref{fig:box} displays the relationship between dominant major 
and median earnings. The boxplots provide a more nuanced understanding of 
patterns between majors. Technical majors have the highest maximum earnings. 
Business and Health Professions feature small interquartile ranges but yield
a high number of outliers. 

\begin{figure}[H]
\centering
\includegraphics[width=1\textwidth]{figures/earnings_scatter.png}
\caption{Median Earnings by University and Dominant Major}
\label{fig:box}
\end{figure}

Figure \ref{fig:scatter} shows the distribution of the top 50
earning universities, revealing significant within-major variation. 
This does suggest that institutional quality matters beyond major choice.
This also clearly shows the dominance of majors in Technology and Social Sciences.

\begin{figure}[H]
\centering
\includegraphics[width=1\textwidth]{figures/T50_universities.png}
\caption{Top 50 Earning Universities and Their Dominant Majors}
\label{fig:scatter}
\end{figure}

\section{Discussion}
These findings have a few possible implications. First, they suggest 
that students seeking to maximize post-graduate 
earnings should not only consider 
individual major choice, but also the broader academic culture and focus of 
their institution. Universities with strong Social Science and Engineering
programs may provide better career networks, employer relationships, 
and signaling value even for students in other fields.

Also, the results highlight potential policy concerns about income 
inequality across different institutions. Students attending 
Agriculture or Liberal Arts-focused colleges face systematic 
lower earnings, which may deepen 
existing disparities in college access.

\subsection{Limitations}
\begin{itemize}
    \item First, this analysis examines correlation, 
not causation. The highest-earning institutions may attract students 
with greater earning 
potential regardless of the education provided. 
    
    \item Also, the earnings measure 
only grades monetary returns and ignores other student values, such as job 
satisfaction. 

    \item The 10-year time period does not properly measure 
long-term earnings, which may drastically differ across fields. For example, 
many Biology majors are likely completing medical school and residency through
their first 10 years after initial enrollment, and 
do not have their earnings properly reflected in this data. 
\end{itemize}

\section{Conclusion}
This project demonstrates a relationship between a university's 
dominant academic major and the median earnings of its graduates. 
Social Science, Engineering, Computer Science, and Transportation-focused 
institutions produce graduates earning substantially more than those from 
Performing Arts or Education-focused schools. These findings confirm the 
importance of institutional characteristics in determining labor market 
outcomes and raise important questions about 
inequality in higher education.

\begin{table}[H]
\centering
\caption{Median Earnings by Dominant Major}
\label{tab:descriptive}
\begin{tabular}{lrr}
\toprule
Dominant Major & Median Earnings & \;\;\; \# of Universities \\
\midrule
Transportation & \$84,131 & 6 \\
Social Sciences & \$73,490 & 89 \\
Engineering & \$72,261 & 48 \\
Computer Science & \$66,757 & 24 \\
Architecture & \$65,668 & 1 \\
Business & \$54,549 & 648 \\
Health Professions & \$54,338 & 325 \\
Communication & \$53,130 & 4 \\
Family \& Consumer Sciences & \$52,485 & 1 \\
Biology & \$52,410 & 67 \\
Engineering Technologies & \$51,867 & 2 \\
Security \& Protective Services & \$51,590 & 16 \\
Philosophy \& Religious Studies & \$51,365 & 4 \\
Psychology & \$51,094 & 31 \\
Visual \& Performing Arts & \$49,131 & 43 \\
Education & \$48,136 & 52 \\
Natural Resources & \$47,626 & 7 \\
Parks \& Recreation & \$47,590 & 19 \\
Multi/Interdisciplinary Studies & \$46,852 & 12 \\
Public Administration & \$45,868 & 8 \\
Mechanics \& Repair & \$44,844 & 2 \\
Liberal Arts & \$44,301 & 47 \\
Personal \& Culinary Services & \$43,418 & 1 \\
Area/Ethnic/Cultural Studies & \$43,101 & 1 \\
Theology & \$42,517 & 24 \\
Agriculture & \$39,605 & 18 \\
\bottomrule
\end{tabular}
\end{table}

\end{document}